\section{Einleitung}
\subsection{Hinweise zu diesem Dokument}
\subsubsection*{Zur Form}
Dieses Dokument liegt in einer hybriden Form vor. \TODO{links}
\subsubsection*{Zur Sprachverwendung}
In diesem Dokument wird das \href{https://de.wikipedia.org/wiki/Generisches_Maskulinum}{\textit{generische Maskulinum}} verwendet, um alle Geschlechter anzusprechen.
Begriffe wie "`Autor"' oder "`Herausgeber"' sind damit als geschlechterneutral zu verstehen.
Die hier gew\"ahlte Form wird in Anlehnung an die aktuellste deutschsprachige Zotero-Version und den zugrunde liegenden Text\footnote{Matthew Gardner und Sara Springfeld, \guillemotright Leitfaden f\"ur bibliografische Angaben\guillemotleft, in: \textit{Musikwissenschaftlichen Arbeiten. Eine Einf\"uhrung}, Kassel 2014.} der Gardner/Springfeld-CSL verwendet.

Diese Entscheidung ist nicht als wertend zu deuten (f\"ur Interessierte liefert Philipp H\"ubl eine umfassende Diskussion zum Thema in seiner Vorlesung \href{https://www.youtube.com/watch?v=yvMGFeQ1gsI}{\textit{Bullshit-Resistenz}}).
\subsubsection*{Ein Wegweiser}
Besonders empfiehlt sich das Lesen des Kapitels \hyperref{haeufige_probleme}{H\"aufige Probleme}.
Eine relativ knappe \"ubersicht des 


