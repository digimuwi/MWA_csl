\section{Guide}\label{guide}
\subsection{Wie installiere ich das Gardner/Springfeld-Zitierschema?}
In diesem Guide wird davon ausgegangen, dass bei Sie das Literaturverwaltungsprogramm \textit{Zotero} bereits installiert haben. Sollte das nicht der Fall sein, finden Sie eine kleine Anleitung dazu auf der \href{https://www.zotero.org/support/installation}{Zotero-Homepage}.

In Zotero werden Zitierstile mit \texttt{.csl}-Dateien definiert.
Die aktuellste Version der \textit{Gardner/Springfeld-CSL} finden Sie im offiziellen \href{https://github.com/digimuwi/MWA_csl}{GitHub-Re\-po\-si\-to\-ry}
unter
\href{https://github.com/digimuwi/MWA_csl/blob/main/musikwissenschaftliches-arbeiten-gardner-springfeld.csl}{musik\-wissen\-schaft\-liches-ar\-bei\-ten-gard\-ner-spring\-feld.csl}.
Dort k\"onnen Sie mit einem Klick auf den Downloadknopf die Datei herunterladen.
\inclc{0.8}{res/github_download.png}
Navigieren Sie in Zotero unter \textit{Bearbeiten $\rightarrow$ Einstellungen} auf den \textit{Zitieren}-Reiter.
Dort sehen Sie eine \"Ubersicht bereits installierter Zitierstile. Mit einem Klick auf den "`+"'-Knopf \"offnet sich ein Dateiauswahldialog.
Navigieren Sie dort zu der gerade runtergeladenen \texttt{.csl}-Datei und \"offnen Sie diese.
\inclc{0.6}{res/zitieren_plus.png}
Das sollte den Zitierstil unter dem Namen "`Musikwissenschaftliches Arbeiten Gardner/Springfeld"' installiert haben (sind bereits einige Zitierstile vorhanden, m\"ussen Sie ggf. etwas in der \"Ubersicht herunterscrollen).
\newpage
\subsection{Wie lege ich eine neue Literatur an?}
Um einen eine Literatur einzutragen, finden Sie zuer
\subsubsection{Wie erkenne ich, um welche Publikationsform es sich handelt?}
\subsubsection{Wie erstelle ich einen neuen Eintrag?}
\subsubsection{Wie kann ich Literatur automatisch importieren?}
Es ist wichtig jede automatische Publikation anhand 
\subsubsection{H\"aufige Probleme}\label{haeufige_probleme}
\subsection{Wie binde ich ein Zitat extern ein?}

