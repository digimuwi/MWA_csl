\section{Guide}\label{guide}
\subsection{Wie installiere ich das Gardner/Springfeld-Zitierschema?}
In diesem Guide wird davon ausgegangen, dass bei Sie das Literaturverwaltungsprogramm \textit{Zotero} bereits installiert haben. Sollte das nicht der Fall sein, finden Sie eine kleine Anleitung dazu auf der \href{https://www.zotero.org/support/installation}{Zotero-Homepage}.

In Zotero werden Zitierstile mit \texttt{.csl}-Dateien definiert.
Die aktuellste Version der \textit{Gardner/Springfeld-CSL} finden Sie im offiziellen \href{https://github.com/digimuwi/MWA_csl}{GitHub-Re\-po\-si\-to\-ry}
unter
\href{https://github.com/digimuwi/MWA_csl/blob/main/musikwissenschaftliches-arbeiten-gardner-springfeld.csl}{musik\-wissen\-schaft\-liches-ar\-bei\-ten-gard\-ner-spring\-feld.csl}.
Dort k\"onnen Sie mit einem Klick auf den Downloadknopf die Datei herunterladen.
\inclc{0.8}{res/github_download.png}
Navigieren Sie in Zotero unter \textit{Bearbeiten $\rightarrow$ Einstellungen} auf den \textit{Zitieren}-Reiter.
Dort sehen Sie eine \"Ubersicht bereits installierter Zitierstile. Mit einem Klick auf den "`+"'-Knopf \"offnet sich ein Dateiauswahldialog.
Navigieren Sie dort zu der gerade runtergeladenen \texttt{.csl}-Datei und \"offnen Sie diese.
\inclc{0.6}{res/zitieren_plus.png}
Das sollte den Zitierstil unter dem Namen "`Musikwissenschaftliches Arbeiten Gardner/Springfeld"' installiert haben (sind bereits einige Zitierstile vorhanden, m\"ussen Sie ggf. etwas in der \"Ubersicht herunterscrollen).
\newpage

\subsection{Wie lege ich eine neue Literatur an?}
Es gibt zwei Arten, einen neuen Eintrag anzulegen: Manuell oder über die 
automatische Importfunktion. In jedem Fall ist es wichtig, dass Sie
die Felder korrekt befüllen, um einen korrekten Nachweis zu erhalten. Hinweise 
dazu finden Sie im folgenden Abschnitt.

\subsubsection{Wie erstelle ich einen neuen Eintrag manuell?}
Nutzen Sie dazu den +-Button in der oberen linken Ecke. Bevor Sie die 
Felder mit Informationen füllen, sollten Sie unbedingt die richtige 
Eintragsart auswählen, da die Felder, die Ihnen zur Verfügung stehen, 
je nach Eintragsart variieren.

\subsubsection{Wie kann ich Literatur automatisch importieren?}
Schneller als das manuelle Anlegen von Literatur ist der automatische Import. 
Am bequemsten funktioniert dies, in dem Sie in Ihrem Browser ein sogenanntes 
\textit{Connector}-Plugin installieren.

\paragraph{Beispiele}
Haben Sie beispielsweise auf \textbf{JSTOR} einen interessanten Aufsatz gefunden,
können Sie mit einem Klick auf das Zoterosymbol einen Zotero-Eintrag 
erstellen, der alle wichtigen Informationen und sogar das Volltext-PDF
bereits enthält.

% hier Screenshot einfügen
Die gleiche Vorgehensweise wird auch für \textbf{RILM} empfohlen. 

In der \textbf{MGG Online} empfiehlt es sich, zunächst auf den Button "Zitieren" 
zu klicken und dann als Zitierstil RIS auszuwählen. Anschließend laden Sie diese 
Datei herunter und importieren sie in Zotero.

\subsubsection{H\"aufige Probleme}\label{haeufige_probleme}
Fast immer müssen Daten, die Sie automatisch importiert haben, nachbereitet werden. 
Häufige Probleme sind:

\begin{itemize}
    \item \textbf{Falsche Publikationsform}
    \item \textbf{Abweichende Trennung von Titel und Untertitel}. Nach Gardner/Springfeld soll hier ein Punkt stehen. Beim Import 
    insbesondere aus OPAC-Katalogen werden Sie hier häufig einen von Leerzeichen umgebenen Doppelpunkt vorfinden.    
    \item \textbf{Kein Title Case}. Englischsprachige Publikationen sollten im Title Case nachgewiesen werden. Sie können 
    das Feld Sprache auf den Wert "en" setzen. Damit wird der Titel beim Zitieren automatisch in Title Case umgewandelt.    
\end{itemize}

\subsection{Wie binde ich ein Zitat extern ein?}
\dots
