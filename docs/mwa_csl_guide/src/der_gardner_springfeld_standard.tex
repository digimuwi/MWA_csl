\section{Einzelhinweise zu verschiedenen Publikationsformen}
Der folgende Abschnitt folgt im Wesentlichen der Struktur des Leitfadens 
von Gardner und Springfeld\footnote{Matthew Gardner und
Sara Springfeld, \emph{Musikwissenschaftlichen Arbeiten. Eine Einf\"uhrung},
Kassel 2014 (B\"arenreiter Studienb\"ucher Musik, 19), S.262f.} und gibt
jeweils zu den einzelnen Publikationsformen Hinweise zur zielführenden Eingabe in Zotero.

\subsection{B\"ucher}

W\"ahlen Sie als Eintragsart \textbf{Buch} aus.

\begin{center}
\begin{tabular}{ll}
\toprule
\textbf{Feld} & \textbf{Anmerkung} \\
\midrule
\textbf{Autor} & Vor- und Nachname sollten getrennt eingegeben werden.\\
 & Mit dem \texttt{+}-Knopf rechts k\"onnen Sie mehrere Autoren angeben. \\
\textbf{Titel} & Achten Sie darauf, Titel und Untertitel mit einem Punkt zu trennen und\\
 & englische Titel im Title Case anzugeben. Sie k\"onnen unter dem \texttt{...}-Knopf\\
 & Titel automatisch in Title Case umformen.\\
\textbf{Ort} & \\
\textbf{Datum} & Tragen Sie hier das Jahr ein.\\
\textbf{Auflage} & Geben Sie hier lediglich eine Zahl ein, so wird diese als\\
 & hochgestellte Zahl vor dem Jahr ausgegeben. Bei weiteren\\
 & Informationen (z.\,B.\,\guillemotright zweite und verbesserte Auflage\guillemotleft) werden\\
 & diese als eigenst\"andiger Text ausgegeben. \\
\bottomrule
\end{tabular}
\end{center}

\subsection{B\"ucher in gez\"ahlten Reihen}

Um eine gez\"ahlte Reihe anzugeben, nutzen Sie die Felder \textbf{Reihe} und \textbf{Nummer der Reihe}. 
Die Reihe wird nur dann ausgegeben, wenn beide Felder gesetzt wurden.

\subsection{Unver\"offentlichte oder online publizierte Hochschulschriften}

\noindent
\textbf{Eintragsart: Dissertation}

\begin{center}
\begin{tabular}{ll}
\toprule
\textbf{Feld} & \textbf{Anmerkung} \\
\midrule
\textbf{Art} & Hier k\"onnen Sie z.\,B.\ \guillemotright Diss\guillemotleft oder \guillemotright Masterarbeit\guillemotleft angeben.\\
\textbf{Universit\"at} & Geben Sie hier die Hochschule oder Universit\"at ein, bei der Arbeit\\
 & eingereicht wurde. \\
\bottomrule
\end{tabular}
\end{center}

\subsection{Reprints}

Reprints lassen sich momentan in Zotero nicht vollst\"andig abbilden. Als Workaround k\"onnen Sie das Feld \emph{Auflage} nutzen, um Informationen zum Reprint einzugeben, z.\,B.\ 
\emph{\guillemotright Faksimile-Nachdruck unter dem Titel Musikalisches Lexikon\guillemotleft}.

\vspace{1.5em}

\section{Unselbstst\"andige Publikationen}

\subsection{Lexikonartikel}

W\"ahlen Sie als Eintragsart \textbf{Enzyklop\"adieartikel}.

\begin{center}
\begin{tabular}{ll}
\toprule
\textbf{Feld} & \textbf{Anmerkung}\\
\midrule
\textbf{Titel} & Geben Sie hier den Titel des Artikels an \dots \\
\textbf{Titel der Enzyklop\"adie} & \dots und hier den Titel des Lexikons bzw. der Enzyklop\"adie.\\
\textbf{Autor und Herausgeber} & Direkt unter dem Titel k\"onnen Sie den Namen des Autors angeben.\\
 & Um den Herausgeber anzugeben, klicken Sie auf den \texttt{+}-Knopf rechts.\\
 & Nun sollte ein weiteres Feld \guillemotright Herausgeber\guillemotleft erscheinen.\\
\bottomrule
\end{tabular}
\end{center}

\subsubsection{Hinweise zu speziellen Lexika}

\paragraph{MGG2}

\begin{center}
\begin{tabular}{ll}
\toprule
\textbf{Feld} & \textbf{Anmerkung}\\
\midrule
\textbf{Auflage} & \guillemotright 2., neubearbeitete Ausgabe\guillemotleft\\
\textbf{Band} & Tragen Sie hier auch den Teil der MGG ein, also z.\,B.\ \guillemotright Sachteil, Bd.\,5\guillemotleft.\\
\textbf{Seiten} & Geben Sie hier die Spaltenzahlen mit dem Zusatz \guillemotright Sp.\guillemotleft ein. Zotero erkennt,\\
 & dass es sich um eine nicht-numerische Seitenzahl handelt und wird den\\
 & ansonsten automatischen Zusatz \guillemotright S. [...]\guillemotleft weglassen.\\
\bottomrule
\end{tabular}
\end{center}

Achten Sie darauf, dass Sie auch beim Zitieren Spalten- und nicht Seitenzahlen angeben.
Den entsprechenden \textit{Locator} können Sie im Zitierdialog auswählen, indem sie 
links neben dem Eingabefeld für die Seitenzahl auf \guillemotright Seite\guillemotleft
klicken. Nun öffnet sich eine Liste, aus der Sie \guillemotright Spalte\guillemotleft 
auswählen können.

\paragraph{NG2}

Bei \textbf{Auflage}: \guillemotright 2., neubearbeitete Ausgabe\guillemotleft eintragen.

\paragraph{Artikel in Onlinelexika (GMO)}

Lassen Sie die Felder \textbf{Ort} und \textbf{Datum} leer und nutzen Sie stattdessen:  
\textbf{Titel}, \textbf{Autor}, \textbf{Titel der Enzyklop\"adie}, \textbf{URL} und \textbf{Heruntergeladen am}.

\subsection{Aufs\"atze in Sammelpublikationen}

Nutzen Sie \textbf{Buchteil} als Eintragsart.

\subsection{Aufs\"atze in Kongressberichten}

Nutzen Sie \textbf{Konferenzpaper} als Eintragsart. Informationen zu Titel und Ort des Kongresses nehmen in den Titel des Konferenzbandes auf, \emph{nicht} in die separat daf\"ur vorgesehenen Felder.

\subsection{Aufs\"atze in Festschriften}

Nutzen Sie die normale Form (s.~o.\ \guillemotright Aufs\"atze in Sammelpublikationen\guillemotleft).

\subsection{Aufs\"atze in wissenschaftlichen Zeitschriften oder Jahrb\"uchern}

Nutzen Sie \textbf{Zeitschriftenartikel} als Eintragsart.

\begin{center}
\begin{tabular}{ll}
\toprule
\textbf{Feld} & \textbf{Anmerkung}\\
\midrule
\textbf{Publikation} & Hier tragen Sie den Namen der Zeitschrift ein \dots\\
\textbf{Band} & \dots hier den Jahrgang ein.\\
\textbf{Ausgabe} & \dots und hier das Heft.\\
\textbf{Zeitschriftenabk\"urzung} & Es ist bei bekannten und oft zitierten Zeitschriften sinnvoll, dieses Feld zu setzen.\\
 & Eine \"Ubersicht \"uber gel\"aufige Abk\"urzungen finden Sie im ersten Band der MGG2.\\
\textbf{DOI} & Bei Onlinezeitschriften sollten Sie hier die DOI des Aufsatzes eintragen,\\
 & also z.\,B.\ \texttt{10.31751/1188}. Beim Zitieren wird automatisch ein\\
 & vollst\"andiger Link erg\"anzt.\\
\textbf{URL} & Sofern eine DOI gesetzt ist, wird dieses Feld beim Zitieren ignoriert.\\
 & Es ist nur relevant, wenn keine DOI verkn\"upft ist.\\
\bottomrule
\end{tabular}
\end{center}

\subsection{Artikel in Zeitungen oder Magazinen}

Eintragsart: \textbf{Zeitungsartikel} (Achtung, nicht Zeitschriftenartikel).

\begin{center}
\begin{tabular}{ll}
\toprule
\textbf{Feld} & \textbf{Anmerkung}\\
\midrule
\textbf{Publikation} & Name der Zeitung\\
\bottomrule
\end{tabular}
\end{center}

\subsection{Dokumente in Briefausgaben oder Dokumentensammlungen}

Nehmen Sie hier nur die Briefausgabe oder Dokumentensammlung selbst in Zotero auf (als Buch). 
Beim Zitieren geben Sie \emph{manuell} die Angaben zum entsprechenden Dokument an und nutzen Zotero 
lediglich f\"ur den zweiten Teil der Angabe, z.\,B.
\begin{verbatim}
Brief von Unger an Graf Esterhazy, Wien, 20. August 1820,
zitiert nach: [an dieser Stelle die Briefausgabe mit Zotero einfügen].
\end{verbatim}
